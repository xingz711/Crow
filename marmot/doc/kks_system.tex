\documentclass[]{article}
\usepackage{graphicx,color}
\usepackage{amssymb}
\usepackage{amsmath,amsthm}

\begin{document}

\tableofcontents 
\section{KKS Model Definitions}

\[
F = [1-h(\eta)]F_a + h(\eta)F_b+wg(\eta) 
\]

\[
c =  [1-h(\eta)]c_a + h(\eta)c_b
\]

\[
\frac{dF_a}{dc_a} = \frac{dF_b}{dc_b}
\]

\section{Materials}
The KKS system uses multiple MOOSE materials to provide values for free energy functions, the switching function $h(\eta)$, and the double well function $g(\eta)$.
Providing these as materials allows the functions to be bundled in a single place, while being used by multiple kernels. Furthermore the automatic differentiation feature 
used in the parsed function materials provides the necessary derivatives at no cost to the developer. The derivatives are stored in material properties and follow a naming scheme 
that is defined in \texttt{KKSBaseMaterial.C}.

\section{Cahn-Hilliard Kernels}

\subsection{\texttt{KKSSplitCHCRes}}
This is the split version. The non-linear variable for this Kernel is the concentration $c$.

\subsubsection{Residual}

In the residual routine we need to calculate the term $R= \frac{dF}{dc}$. We exploit the KKS identity $\frac{dF}{dc}=\frac{dF_a}{dc_a}=\frac{dF_b}{dc_b}$ and arbitrarily use the a-phase instead.
\[
R = \frac{dF_a}{dc_a}
\]

\subsubsection{Jacobian}
Starting from the original residual with $\frac{dF}{dc}$. Then
\begin{eqnarray*}
J &=& \phi_j \frac{d}{dc} \frac{dF}{dc}\\
  &=& \phi_j  \frac{d^2F}{dc^2} \quad,\quad \text{with (29) from \cite{KKS}}\\
  &=& \phi_j  \frac{\frac{d^2F_b}{dc_b^2}\frac{d^2F_a}{dc_a^2}}{  [1-h(\eta)]\frac{d^2F_b}{dc_b^2}+h(\eta)\frac{d^2F_a}{dc_a^2} }\\
\end{eqnarray*}


\subsection{\texttt{KKSCHBulk}}
This is the non-split version, which is not fully implemented. The non-linear variable for this Kernel is the concentration $c$.


\subsubsection{Residual}

In the residual routine we need to calculate the term $R=\nabla \frac{dF}{dc}$. We exploit the KKS identity $\frac{dF}{dc}=\frac{dF_a}{dc_a}=\frac{dF_b}{dc_b}$ and arbitrarily use the a-phase instead.
The gradient can be calculated through the chain rule (note that $F_a(c_a, p_1,p_2,\dots,p_n)$ is potentially a function of many variables)
\[
R = \nabla \frac{dF_a}{dc_a} = \frac{d^2F_a}{dc_a^2}\nabla c_a + \sum_i \frac{d^2F_a}{dc_adp_i}\nabla p_i 
\] 

With $a = \{c_a, p_1,p_2,\dots,p_n\}$ being the vector of all arguments to $F_a$ this simplifies to 
\[
R=\sum_i \frac{d^2F_a}{dc_ada_i}\nabla a_i  = \sum_i R_i \nabla a_i
\]

using $R_i$ as a shorthand for the term $\frac{d^2F_a}{dc_ada_i}$ (and represented in the code as the array \texttt{\_second\_derivatives[i]}). We do have access to the gradients of $a_i$ through MOOSE (stored in \texttt{\_grad\_args[i]}).

\subsubsection{Jacobian}

The calculation of the Jacobian involves the derivative of the Residual term $R$ w.r.t. the individual coefficients $u_j$ of all parameters of $F_a$. Here $u$ can stand for 
any variable $a_i$.
\[
\frac{dR}{du_j} = \sum_i \left[ \frac d{du_j} R_i\nabla a_i \right] = \sum_i \left[  R_i\frac{d\nabla a_i}{du_j} + \nabla a_i \sum_k \frac {dR_i}{da_k}\frac{da_k}{du_j} \right] 
\]

In the code $u$ is given by \texttt{jvar} for the off diagonal case, and $c$ (not $c_a$ or $c_b$!) in the on diagonal case. 

\paragraph{Off-diagonal}

Let's focus on off diagonal first. Here $\frac{da_k}{du_j}$ is zero, if \texttt{jvar} is not equal $k$. Allowing us to remove the sum over k and replace it with the single non-zero summand
\[
\frac{dR}{du_j} = \sum_i \left[  R_i\frac{d\nabla a_i}{du_j} + \nabla a_i \frac {dR_i}{da_\text{jvar}}\frac{da_\text{jvar}}{du_j} \right] 
\]

In the first term in the square brackets the derivative $\frac{d\nabla a_i}{du_j}$ is only non-zero if $i$ is \texttt{jvar}. We can therefore pull this term out of the sum.
\[
\frac{dR}{du_j} = R_\text{jvar}\frac{d\nabla a_\text{jvar}}{du_j} + \sum_i  \nabla a_i \frac {dR_i}{da_\text{jvar}}\frac{da_\text{jvar}}{du_j} 
\]

With the rules for $\frac d{du_j}$ derivatives we get
\[
 R_\text{jvar} \nabla \phi_j + \sum_i \nabla a_i \frac {dR_i}{da_\text{jvar}} \phi_j 
\]

where $j$ is \texttt{\_j} in the code.

\paragraph{On-diagonal}

For the on diagonal terms we look at the derivative w.r.t. the components of the non-linear variable $c$ of this kernel. Note, that $F_a$ is only indirectly a function of $c$. We assume the dependence is given through $c(c_a)$. The chain rule will thus yield terms of this form  
\[
\frac{dc_a}{dc} = \frac{\frac{d^2F_b}{dc_b^2}}{[1-h(\eta)]\frac{d^2F_b}{dc_b^2}+h(\eta)\frac{d^2F_a}{dc_a^2}},
\]
%\[
%\frac{\partial c_b}{c\partial c} = \frac{\frac{d^2F_a}{dc_b^2}}{[1-h(\eta)]\frac{d^2F_b}{dc_b^2}+h(\eta)\frac{d^2F_a}{dc_a^2}}.
%\]

which is given as equation (23) in \cite{KKS}. Following the off-diagonal  derivation we get
\[
\frac{d^2F_a}{dc_a^2}\frac{dc_a}{dc} \nabla \phi_j + \sum_i \nabla a_i \frac {dR_i}{dc_a} \frac{dc_a}{dc} \phi_j 
\]

\paragraph{On-diagonal second approach}
Let's get back to the original residual with $\frac{dF}{dc}$. Then
\begin{eqnarray*}
J &=& \phi_j \frac{d}{dc} \nabla \frac{dF}{dc}\\
  &=& \phi_j  \nabla \frac{d^2F}{dc^2} \quad,\quad \text{with (29) from \cite{KKS}}\\
  &=& \phi_j  \nabla \frac{\frac{d^2F_b}{dc_b^2}\frac{d^2F_a}{dc_a^2}}{  [1-h(\eta)]\frac{d^2F_b}{dc_b^2}+h(\eta)\frac{d^2F_a}{dc_a^2} }\\
\end{eqnarray*}

\section{Allen-Cahn  Kernels}

For the bulk Allen-Cahn residual we need to calculate the term 
\begin{eqnarray*}
R=\frac{dF}{d\eta}&=&\frac{d}{d\eta}\left([1-h(\eta)]F_a + h(\eta)F_b+wg(\eta) \right)\\
&=& \frac{dF_a}{d\eta}+\frac{d}{d\eta}\left(h(\eta)F_b-h(\eta)F_a + wg(\eta)\right)\\
&=& \frac{dF_a}{d\eta}+F_b\frac{dh}{d\eta}-F_a\frac{dh}{d\eta}+h(\eta)\frac{dF_b}{d\eta}-h(\eta)\frac{dF_a}{d\eta} +w\frac{dg}{d\eta}\\
&=& \frac{dF_a}{d\eta}+\frac{dh}{d\eta}(F_b-F_a)+h(\eta)\left(\frac{dF_b}{d\eta}-\frac{dF_a}{d\eta}\right) +w\frac{dg}{d\eta}\\
&=& \frac{dh}{d\eta}(F_b-F_a) + \underbrace{[1-h(\eta)]\frac{dF_a}{d\eta} + h(\eta)\frac{dF_b}{d\eta}}_{\text{chain rule term}} +w\frac{dg}{d\eta}\\
&=& \frac{dh}{d\eta}(F_b-F_a) + [1-h(\eta)]\frac{dF_a}{dc_a}\frac{dc_a}{d\eta} + h(\eta)\frac{dF_b}{dc_b}\frac{dc_b}{d\eta} +w\frac{dg}{d\eta}\\
&=& \frac{dh}{d\eta}(F_b-F_a) + \frac{dF_a}{dc_a}\left([1-h(\eta)]\frac{dc_a}{d\eta} + h(\eta)\frac{dc_b}{d\eta}\right) +w\frac{dg}{d\eta}
\end{eqnarray*}

The \emph{chain rule term} results from the fact that $c_a$ and $c_b$ are dependent on $\eta$ (see eqs. (25) and (26) in \cite{KKS}).
Setting $\lambda = [1-h(\eta)]\frac{d^2F_b}{dc_b^2}+h(\eta)\frac{d^2F_a}{dc_a^2}$ we get
\begin{eqnarray*}
\frac{dc_a}{d\eta} &=& \frac1\lambda \frac{dh}{d\eta}(c_a-c_b)\frac{d^2F_b}{dc_b^2}\\
\frac{dc_b}{d\eta} &=& \frac1\lambda \frac{dh}{d\eta}(c_a-c_b)\frac{d^2F_a}{dc_a^2}
\end{eqnarray*}

Substituting this in we get
\begin{eqnarray*}
R &=& \frac{dh}{d\eta}(F_b-F_a) + \frac{dh}{d\eta}\frac{dF_a}{dc_a}(c_a-c_b)\frac1\lambda\underbrace{\left([1-h(\eta)] \frac{d^2F_b}{dc_b^2} + h(\eta)\frac{d^2F_a}{dc_a^2}\right)}_{=\lambda} +w\frac{dg}{d\eta}
\end{eqnarray*}

This simplifies to
\[
R=-\frac{dh}{d\eta} \left(F_a-F_b-\frac{dF_a}{dc_a}(c_a-c_b)\right) + w\frac{dg}{d\eta}.
\]

We split this residual into two kernels to allow for multiple phase concentrations in a multi component system:

\subsection{\texttt{KKSACBulkF}}

\subsubsection{Residual}
\[
R=-\frac{dh}{d\eta}(F_a-F_b)+w\frac{dg}{d\eta}.
\]

\subsubsection{Jacobian}

\paragraph{On-diagonal}
We are looking for the $\frac d{du_j}$ derivative of R, where $u\equiv\eta$. We need to apply the chain rule and will again only keep terms with the $\frac{d}{d\eta}\frac{d\eta}{du_j}=\frac{d}{d\eta}\phi_j$ derivative. Note that $F_a$ and $F_b$ indirectly are functions of $\eta$.
\[
-\frac{d^2h}{d\eta^2}\phi_j(F_a-F_b) \underbrace{-\frac{dh}{d\eta}\left(\frac{dF_a}{d\eta}-\frac{dF_b}{d\eta}\right)\phi_j}_{\text{handled in \texttt{KKSACBulkC}}} + w\frac{d^2g}{d\eta^2}\phi_j.
\]

Part of this Jacobian is handled in \texttt{KKSACBulkC}, as the $\frac F{d\eta}$ derivatives require knowledge of the phase concentration variables (via chain rule).

\subsection{\texttt{KKSACBulkC}}

\subsubsection{Residual}
\begin{eqnarray*}
R&=&-\frac{dh}{d\eta}\left(-\frac{dF_a}{dc_a}(c_a-c_b)\right)\\
  &=&\frac{dh}{d\eta}\frac{dF_a}{dc_a}(c_a-c_b)
\end{eqnarray*}

\subsubsection{Jacobian}

\paragraph{On-diagonal}
We are looking for the $\frac d{du_j}$ derivative of R, where $u\equiv\eta$. We need to apply the chain rule and will again only keep terms with the $\frac{d}{d\eta}\frac{d\eta}{du_j}=\frac{d}{d\eta}\phi_j$ derivative. Note that $F_a$ and $F_b$ indirectly are functions of $\eta$.
\begin{eqnarray*}
J_1 &=& \frac{d^2h}{d\eta^2}\phi_j\frac{dF_a}{dc_a}(c_a-c_b) + \frac{dh}{d\eta}\left(\frac{d}{d\eta}\left[ \frac{dF_a}{dc_a}(c_a-c_b)\right]\right)\phi_j\\
&=& \frac{d^2h}{d\eta^2}\phi_j\frac{dF_a}{dc_a}(c_a-c_b) + \frac{dh}{d\eta}\left( (c_a-c_b)\frac{d}{d\eta}\frac{dF_a}{dc_a} + \frac{dF_a}{dc_a}\frac{d}{d\eta}(c_a-c_b) \right)\phi_j\\
&=& \frac{d^2h}{d\eta^2}\phi_j\frac{dF_a}{dc_a}(c_a-c_b) + \frac{dh}{d\eta}\left( (c_a-c_b)\frac{d^2F_a}{dc_a^2}\frac{dc_a}{d\eta} + \frac{dF_a}{dc_a}\left[\frac{dc_a}{d\eta}-\frac{dc_b}{d\eta}\right] \right)\phi_j\\
&=& \frac{d^2h}{d\eta^2}\phi_j\frac{dF_a}{dc_a}(c_a-c_b) + \frac{dh}{d\eta}\left( (c_a-c_b)\frac{d^2F_a}{dc_a^2}\frac{dc_a}{d\eta}\right)\phi_j + \frac{dh}{d\eta}\left(\frac{dF_a}{dc_a}\frac{dc_a}{d\eta}-\frac{dF_b}{dc_b}\frac{dc_b}{d\eta} \right)\phi_j
\end{eqnarray*}

Onto this we have to add the term we pick up from \texttt{KKSACBulkF}
\begin{eqnarray*}
J_2 &=& -\frac{dh}{d\eta}\left(\frac{dF_a}{d\eta}-\frac{dF_b}{d\eta}\right)\phi_j\\
&=&-\frac{dh}{d\eta}\left(\frac{dF_a}{dc_a}\frac{dc_a}{d\eta}-\frac{dF_b}{dc_b}\frac{dc_b}{d\eta}\right)\phi_j
\end{eqnarray*}

Which cancels the last term in $J_1$ (note the chemical potential equality is used), leaving the Jacobian 
\[
J=\frac{d^2h}{d\eta^2}\phi_j\frac{dF_a}{dc_a}(c_a-c_b) + \frac{dh}{d\eta}\left( (c_a-c_b)\frac{d^2F_a}{dc_a^2}\frac{dc_a}{d\eta}\right)\phi_j 
\]

TODO: investigate if we are missing cross terms in case of multi-component systems ($\eta$ derivative of $F_a$ may have more chain rule terms)!

\paragraph{Off-diagonal}
%This simplifies greatly, as neither $g$ nor $h$ are functions of any variable besides $\eta$. The only term that remains is
%\[
%\frac{dh}{d\eta}\left(\frac{dF_a}{du_\text{jvar}}-\frac{dF_b}{du_\text{jvar}}\right)\phi_j
%\] 

\section{Constraint Kernels}

\subsection{\texttt{KKSPhaseChemicalPotential}}

This Kernel enforces the point wise equality of the phase chemical potentials
\[
\frac{dF_a}{dc_a}=\frac{dF_b}{dc_b}
\]
The non-linear variable of this Kernel is $c_a$.

\subsubsection{Residual}
\[
R=\frac{dF_a}{dc_a} - \frac{dF_b}{dc_b}
\]

\subsubsection{Jacobian}

For the Jacobian we need to calculate 
\[
J=\frac d{du_j}\left( \frac{dF_a}{dc_a} - \frac{dF_b}{dc_b} \right).
\]

With $q$ the union of the argument vectors of $F_a$ and $F_b$ (represented in the code by \texttt{\_coupled\_moose\_vars[]}) we get
\[
\sum_i \left( \frac{d^2F_a}{dc_aq_i}\frac{dq_i}{du_j} + \frac{d^2F_b}{dc_bq_i}\frac{dq_i}{du_j} \right).
\]

Again the $\frac{dq_i}{du_j}$ is non-zero only if $u\equiv q_i$, which is the case if $q_i$ is the argument selected through \texttt{jvar}.
\[
\frac{d^2F_a}{dc_aq_\text{jvar}}\phi_j + \frac{d^2F_b}{dc_bq_\text{jvar}}\phi_j.
\]

Note that in the code \texttt{jvar} is not an index into \texttt{\_coupled\_moose\_vars[]} but has to be resolved through the \texttt{\_jvar\_map}.

\subsection{\texttt{KKSPhaseConcentrations}}
This Kernel enforces the split of the concentration into the phase concentrations, weighted by the switching function. The non-linear variable of this Kernel is $c_b$.
\[
c = [1-h(\eta)]c_a+h(\eta)c_b
\]

\subsubsection{Residual}
\[
R=[1-h(\eta)]c_a + h(\eta)c_b - c
\]

\subsubsection{Jacobian}
\[
J=
\]






\subsection{\texttt{KKSACBulk}}

\subsubsection{Residual}

In the residual routine we need to calculate the term 
\begin{eqnarray*}
R=\frac{dF}{d\eta}&=&\frac{d}{d\eta}\left([1-h(\eta)]F_a + h(\eta)F_b+wg(\eta) \right)\\
&=& \frac{dF_a}{d\eta}+\frac{d}{d\eta}\left(h(\eta)F_b-h(\eta)F_a + wg(\eta)\right)\\
&=& \frac{dF_a}{d\eta}+F_b\frac{dh}{d\eta}-F_a\frac{dh}{d\eta}+h(\eta)\frac{dF_b}{d\eta}-h(\eta)\frac{dF_a}{d\eta} +w\frac{dg}{d\eta}\\
&=& \frac{dF_a}{d\eta}+\frac{dh}{d\eta}(F_b-F_a)+h(\eta)\left(\frac{dF_b}{d\eta}-\frac{dF_a}{d\eta}\right) +w\frac{dg}{d\eta}\\
&=& \frac{dh}{d\eta}(F_b-F_a) + \underbrace{[1-h(\eta)]\frac{dF_a}{d\eta} + h(\eta)\frac{dF_b}{d\eta}}_{\text{chain rule term}} +w\frac{dg}{d\eta}\\
&=& \frac{dh}{d\eta}(F_b-F_a) + [1-h(\eta)]\frac{dF_a}{dc_a}\frac{dc_a}{d\eta} + h(\eta)\frac{dF_b}{dc_b}\frac{dc_b}{d\eta} +w\frac{dg}{d\eta}\\
&=& \frac{dh}{d\eta}(F_b-F_a) + \frac{dF_a}{dc_a}\left([1-h(\eta)]\frac{dc_a}{d\eta} + h(\eta)\frac{dc_b}{d\eta}\right) +w\frac{dg}{d\eta}
\end{eqnarray*}

The \emph{chain rule term} results from the fact that $c_a$ and $c_b$ are dependent on $\eta$ (see eqs. (25) and (26) in \cite{KKS}).
Setting $\lambda = [1-h(\eta)]\frac{d^2F_b}{dc_b^2}+h(\eta)\frac{d^2F_a}{dc_a^2}$ we get
\begin{eqnarray*}
\frac{dc_a}{d\eta} &=& \frac1\lambda \frac{dh}{d\eta}(c_a-c_b)\frac{d^2F_b}{dc_b^2}\\
\frac{dc_b}{d\eta} &=& \frac1\lambda \frac{dh}{d\eta}(c_a-c_b)\frac{d^2F_a}{dc_a^2}
\end{eqnarray*}

Substituting this in we get
\begin{eqnarray*}
R &=& \frac{dh}{d\eta}(F_b-F_a) + \frac{dh}{d\eta}\frac{dF_a}{dc_a}(c_a-c_b)\frac1\lambda\underbrace{\left([1-h(\eta)] \frac{d^2F_b}{dc_b^2} + h(\eta)\frac{d^2F_a}{dc_a^2}\right)}_{=\lambda} +w\frac{dg}{d\eta}
\end{eqnarray*}

This simplifies to
\[
R=-\frac{dh}{d\eta}\underbrace{\left(F_a-F_b-\frac{dF_a}{dc_a}(c_a-c_b)\right)}_{\text{\texttt{A1} in the code}}+w\frac{dg}{d\eta}.
\]

\subsubsection{Jacobian}

\paragraph{On-diagonal}
We are looking for the $\frac d{du_j}$ derivative of R, where $u\equiv\eta$. We need to apply the chain rule and will again only keep terms with the $\frac{d}{d\eta}\frac{d\eta}{du_j}=\frac{d}{d\eta}\phi_j$ derivative. Note that $F_a$ and $F_b$ indirectly are functions of $\eta$.
\[
\frac{d^2h}{d\eta^2}\phi_j(F_a-F_b-\frac{dF_a}{dc_a}(c_a-c_b)) + \frac{dh}{d\eta}\left(\frac{dF_a}{d\eta}-\frac{dF_b}{d\eta}-\frac{d}{d\eta}\left[ \frac{dF_a}{dc_a}(c_a-c_b)\right]\right)\phi_j + w\frac{d^2g}{d\eta^2}\phi_j.
\]

\paragraph{Off-diagonal}
This simplifies greatly, as neither $g$ nor $h$ are functions of any variable besides $\eta$. The only term that remains is
\[
\frac{dh}{d\eta}\left(\frac{dF_a}{du_\text{jvar}}-\frac{dF_b}{du_\text{jvar}}\right)\phi_j
\] 


\begin{thebibliography}{9}
\bibitem{KKS} S.G. Kim, W.T. Kim, and T. Suzuki, Phys. Rev. E 60 (1999) 7186
\end{thebibliography}

\end{document}